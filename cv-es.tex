%%%%%%%%%%%%%%%%%%%%%%%%%%%%%%%%%%%%%%%%%
% Twenty Seconds Resume/CV
% LaTeX Template
% Version 1.0 (14/7/16)
%
% Original author:
% Carmine Spagnuolo (cspagnuolo@unisa.it) with major modifications by 
% Vel (vel@LaTeXTemplates.com) and Harsh (harsh.gadgil@gmail.com)
%
% License:
% The MIT License (see included LICENSE file)
%
%%%%%%%%%%%%%%%%%%%%%%%%%%%%%%%%%%%%%%%%%

%----------------------------------------------------------------------------------------
%	PACKAGES AND OTHER DOCUMENT CONFIGURATIONS
%----------------------------------------------------------------------------------------

\documentclass[letterpaper]{twentysecondcv} % a4paper for A4
\usepackage[spanish]{babel}
\usepackage{xunicode,xltxtra,url,parskip} % Formatting packages
% Command for printing skill overview bubbles
\newcommand\skills{ 
~
	\smartdiagram[bubble diagram]{
        \textbf{Ingeriría} \\\textbf{Física},
        \textbf{Abstracción}\\\textbf{Matemática},
        \textbf{Conducción}\\\textbf{Autónoma},
        \textbf{Simulaciones}\\\textbf{Dinámicas},
        \textbf{Machine}\\\textbf{Learning},
        \textbf{Desarrollo}\\\textbf{Software},
        \textbf{Análisis}\\\textbf{Estadística}
    }
}

% Programming skill bars
\programming{{HTML5 $\textbullet$ Java/ 1},{gnuplot $\textbullet$ javascript/ 2},{Matlab $\textbullet$ ADTF $\textbullet$ CUDA $\textbullet$ OpenMP/ 3.5},{C $\textbullet$ C++  $\textbullet$ \LaTeX $\textbullet$ python  $\textbullet$ Mathematica/ 5}}

\languages{{Francés $\textbullet$ Polaco /1},{Español/ 5},{Portugués / 6},{Inglés / 6}}

\projects{
\textbf{DecAR} - An augmented reality interior decoration app for Android \\
        \textbf{CIS*6320} - An implementation of the Bicubic interpolation algorithm in C++
        \textbf{CIS*6650} - A comparative statistical study of SVM kernels, and number of hidden layers in ANN, on 5 UCI datasets
        \textbf{CIS*6660} - A data linkage project to integrate Canada's WWI casualties and 1901 Canadian census using an SVM
        \textbf{CIS*6650} - A statistical study of spurious correlations, such as correlating beer production with election outcome
}



%----------------------------------------------------------------------------------------
%	 PERSONAL INFORMATION
%----------------------------------------------------------------------------------------
% If you don't need one or more of the below, just remove the content leaving the command, e.g. \cvnumberphone{}

\cvname{Rui Marques} % Your name
\cvjobtitle{ Ingeniero Físico } % Job
% title/career
\profilepic{photo.png}
\cvlinkedin{in/ruiferreiramarques}
\cvgithub{github.com/ruifm}
\cvnumberphone{\href{tel:+34663559607}{(+34) 663 559 607}} % Phone number
\cvsite{hgadgil.com} % Personal website
\cvmail{ruifmarques@pm.me} % Email address
\cvskype{ruifm75969}
\cvaddress{\href{https://www.google.es/maps/place/R\%C3\%BAa+\%C3\%81ngel+Senra,+29,+15007+A+Coru\%C3\%B1a/\@43.3548567,-8.4109767,17z/data=!3m1!4b1!4m5!3m4!1s0xd2e7c906066d703:0x5930f780929bf00b!8m2!3d43.3548567!4d-8.408788?hl=en}{Rúa Ángel Senra 29, 8E (15007 A Coruña)}}


\aboutme{Desarrollador de LiDAR en ADAS y Estudiante de último año de Ingeniería de Física. Buscando desafíos en Tecnología en A Coruña.

El enfoque extremo, la concentración, la capacidad de consumir grandes cantidades de información, hacer frente a un plazo cercano son mis mejores y más útiles cualidades, adquiridos en un ambiente altamente competitivo.}

%----------------------------------------------------------------------------------------

\begin{document}

\makeprofile % Print the sidebar

%----------------------------------------------------------------------------------------
%    EXPERIENCE
%----------------------------------------------------------------------------------------

\section{Experiencia}

\begin{twenty} % Environment for a list with descriptions
\twentyitem
        {Ene 2018}
        {Presente}
        {Desarrollador de Software de ADAS}
        {\href{https://www.ctag.com/}{CTAG (Centro Tecnológico de Automoción de Galicia)}}
        {Vigo, España}
        {\begin{itemize}
        \item Trabajé al principio en un proyecto interno como desarrollador del sistema LiDAR de un coche autónomo.
        \item Debido a la rapidez con que terminé el proyecto, me transferirán para un equipo que trabaja directamente para un cliente.
        \item Trabajé con 2 Metodologías Ágiles: SCRUM y Kanban.
        \item Desarrollé software en un entorno colaborativo de equipo.
        \item Cogí la base de código entera de un compañero que estaba de salida.
        \item Gestioné, propuse y revisé Requisitos de Sistema y de Software.
        \item Desarrollé y mantuve software cuando pedido por el Cliente siempre con cualidad y antes del plazo.
        \item Mantuve una comunicación eficiente con el Cliente.
        \item Realicé verificaciones semanales de software: sanity checks, runtime checks, code coverage tests, static tests (code linting against MISRA-C++) y las reporté al Cliente.
        \item \textbf{Herramientas:} C++, ADTF, Eclipse, Qt, Qt-creator, python, matplotlib, numpy, bash, Matlab, Redmine, git, Octave, MSVS, doxygen, cmake, CANAlyzer, CANoe, Wireshark, Makefile, JIRA, Serena, DOORS, Google Docs, html, xml, \LaTeX
        \item \textbf{Formaciones:} Ciclo de Vida V, Gestión de Requisitos, Automotive Spice, ISO26262, DOORS, SCRUM 
        \end{itemize}}
        \\
    \twentyitem
        {Sep 2017}
        {Oct 2017}
        {Agente Comercial}
        {M\&S Servicios y Marketing}
        {A Coruña, España}
        {Responsable de traer un flujo constante de nuevos negocios rentables en el sector de la energía proporcionando soluciones de vanguardia a potenciales clientes.}
    \\   
    \twentyitem
        {Ago 2012}
        {Sep 2012}
        {Becario en Física de Partículas}
        {\href{https://www.lip.pt}{LIP (Laboratorio Portugués de Física de Partículas)}}
        {Lisboa, Portugal}
        {Trabajé en datos de rayos cósmicos y simulaciones por ordenador del \emph{Observatorio Pierre Auger}. Supervisor: Pedro Abreu (\href{mailto:abreu@lip.pt}{abreu@lip.pt}))
        }
\end{twenty}

\section{Investigación}
\begin{twenty}
    \twentyitem
        {Feb 2017}
        {Ene 2018}
        {MSc.  Investigador Asistente}
        {\href{https://centra.tecnico.ulisboa.pt/network/grit/team/}{GRIT-CENTRA}}
        {Lisboa, Portugal}
        {
        \textbf{Tesis}: Gravitones masivos y discontinuidades vDZV
        {\begin{itemize}
        \item Investigué una extensión de la Teoría de Relatividad General de Einstein en que el gravitón tiene una masa no nula y exploré sus consequencias en resultados experimentales como ondas gravitacionales.
        \item \textbf{Herramientas}: Python, scikit-learn, \LaTeX, matplotlib, Mathematica, gnuplot
        \end{itemize}}
        }
\end{twenty}

%----------------------------------------------------------------------------------------
%	 EDUCATION
%----------------------------------------------------------------------------------------
\section{Formación Académica}

\begin{twenty} % Environment for a list with descriptions
	\twentyitem
    	{Sep 2016}
        {Jun 2017}
        {Erasmus \@ \textsc{Física Teórica}}
        {\href{http://web.science.uu.nl/itf/}{ITF, Universidad de Utrecht}}
        {Utrecht, Holanda}
        {Estudié tópicos de información cuántica.
        Referencia Académica: Enrico Pajer (\href{mailto:e.pajer@uu.nl}{e.pajer@uu.nl}) | \normalsize \textsc{Media}: 8/10}
	\twentyitem
    	{Sep 2012}
		{Feb 2018}
        {MSc in \textsc{\href{https://www.youtube.com/watch?v=0eoa0f5nVA0}{Ingeniería Física}} \faYoutubePlay}
        {\href{https://www.youtube.com/watch?v=EGue8EwE3mI}{Instituto Superior Tecnico} \faYoutubePlay}
        {Lisboa, Portugal}
        {Especialización en Física de Alta Energía}
    \twentyitem
        {Sep 2009}
        {Feb 2012}
        {Escuela Secundária \@ \textsc{Ciencias Naturales}}
        {Escola Secundária D. Pedro V}
        {Lisboa, Portugal}
        {Física y Matemática: 20/20, Inglés 19/20 | mejor estudiante de 2012}
	%\twentyitem{<dates>}{<title>}{<organization>}{<location>}{<description>}
\end{twenty}

\newpage

\continuesidebar

\section{Becas y Certificados}
\begin{twenty} % Environment for a list with descriptions
    \twentyitem
        {Oct 2017}
        {}
        {Redes Neuronales y Aprendizaje Profundo por deeplearning.ai}
        {\href{https://www.coursera.org/account/accomplishments/records/CF342FNNVUXY}{Coursera}}
        {}
        {}
    \twentyitem
        {Oct 2017}
        {}
        {Mejora de Redes Neuronales Profundas: Optimización, Regularización y Optimización de Hiperparesúmenes por deeplearning.ai}
        {\href{https://www.coursera.org/account/accomplishments/records/MQVCJFJ849GK}{Coursera}}
        {}
        {}
    \twentyitem
        {May 2012}
        {}
        {Finalista Nacional}
        {\href{http://www.sp-astronomia.pt/olimpiadas}{Olimpiadas de Astronomía e Astrofísica Portuguesas}}
        {São Miguel, Azores, Portugal}
        {}
    \twentyitem
        {Ene 2012}
        {May 2012}
        {Escuela de física pre-universitaria avanzada}
        {\href{http://quark.fis.uc.pt/}{Proyecto Quark!}}
        {Coimbra, Portugal}
        {}
    %\twentyitem{<dates>}{<title>}{<organization>}{<location>}{<description>}
\end{twenty}
\vspace{-0.5cm}
\section{Proyectos de Programación}

\begin{twenty} % Environment for a list with descriptions
    \twentyitem
        {Oct 2017}
        {Presente}
        {Space Out}
        {\href{https://github.com/ruifm/space-out}{\faGithub github.com/ruifm/space-out}}
        {}
        {Versión 1vs1 en Pygame del clásico juego de arcade 'Asteroids'. Destinado para el entrenamiento de una red neuronal de refuerzo de aprendizaje para trabajar como un oponente de IA.}
    \twentyitem
        {Dec 2016}
        {Ene 2017}
        {Entropía de Enredo}
        {\href{https://github.com/ruifm/ed-triangular}{\faGithub github.com/ruifm/ed-triangular}}
        {}
        {Cuaderno de Mathematica para la diagonalización exacta y cálculo de entropía de enredo en un enrejado de spin triangular.}
    \twentyitem
        {May 2015}
        {}
        {Oort Cloud}
        {\href{https://github.com/ruifm/oort}{\faGithub github.com/ruifm/oort}}
        {}
        {Una versión de JavaScript que utiliza el framework Phaser del clásico juego de arcade 'Asteroids', modificado al darle un toque de 'flappy bird', i.e. un juego interminable. Alojado aquí: \href{www.xente.mundo-r.com/20624313W0001/index.html}{xente.mundo-r.com/20624313W0001/index.html}}
    \twentyitem
        {May 2014}
        {}
        {Antifitter}
        {\href{https://github.com/ruifm/antifitter}{\faGithub github.com/ruifm/antifitter}}
        {}
        {Programa en C ++ que produce datos experimentales falsos para ajustarse a una determinada función y traza automáticamente.}
     \twentyitem
        {Dec 2013}
        {Ene 2014}
        {Gross-Pitaevskii Simulator}
        {\href{https://github.com/ruifm/gross-pitaevskii}{\faGithub github.com/ruifm/gross-pitaevskii}}
        {}
        {Simulación de la gráfica de densidad de color de la ecuación de Gross-Pitaevskii aplicada a un condensado de Bose-Einstein usando C++, OpenMP y CUDA. Resultado: \href{https://youtu.be/V091IqIRV4c}{youtu.be/V091IqIRV4c}}
    \twentyitem
        {Dec 2012}
        {Ene 2013}
        {Coloumbian Simulator}
        {\href{https://github.com/ruifm/charges}{\faGithub github.com/ruifm/charges}}
        {}
        {Simulador de la fuerza de Coulomb en cargos escrito en C con GTK+.}
    \twentyitem
        {Nov 2012}
        {}
        {Atkin's Sieve in C}
        {\href{https://github.com/ruifm/atkin}{\faGithub github.com/ruifm/atkin}}
        {}
        {Implementación de Atkin para encontrar números primos en C con muchas características.}       
    %\twentyitem{<dates>}{<title>}{<organization>}{<location>}{<description>}
\end{twenty}

\section{Proyectos y Afiliaciones}

\begin{twenty}
    \twentyitem
        {Feb 2014}
        {}
        {Co-anfitrión y organizador}
        {\href{http://jornadasdefisica.nfist.pt/index.html}{Jornadas de Ingeniería Física}}
        {Lisboa, Portugal}
        {Un evento de 3 días con seminarios de física e ingeniería con investigadores y potenciales empleadores. Dirigí un grupo de trabajo de 7 personas para organizar este asombroso evento.}
    \twentyitem
        {Sep 2013}
        {Sep 2014}
        {Animador de Ciencias}
        {\href{http://festadoavante.pcp.pt/2016/}{Festa do Avante}}
        {Amora, Lisboa, Portugal}
        {He realizado experimentos de física en vivo al público durante el festival de verano del \emph{Avante}.}
    \twentyitem
        {2012}
        {2015}
        {Miembro de la Junta Directiva}
        {\href{http://nfist.pt}{NFIST} (Club de Física de IST)}
        {Lisboa, Portugal}
        {Una organización dinámica y productiva sin fines de lucro con un enorme alcance científico.Su objetivo principal es crear conciencia pública sobre la belleza y la omnipresencia de la física en nuestra vida cotidiana. Enseñé activamente ciencias en escuelas públicas, museos y y otros eventos.}
    \twentyitem
        {Sep 2012}
        {}
        {Miembro del personal}
        {\href{http://www.lip.pt/cmsweek2012/}{CMS Week 2012}}
        {Lisboa, Portugal}
        {Un físico de LIP (ex supervisor) me invitó a formar parte del comité organizador y responsable de organizar el evento. Tuve la oportunidad de conocer físicos de renombre en todo el mundo.}
\end{twenty}



\end{document} 
