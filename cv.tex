\documentclass[letterpaper]{twentysecondcv} % a4paper for A4
\usepackage{common}
\newif\ifen{}
\newif\ifspa{}

\newcommand{\en}[1]{\ifen#1\fi}
\newcommand{\spa}[1]{\ifspa#1\fi}

\entrue{}
% \spatrue{}

% Command for printing skill overview bubbles
\newcommand\skills{
	\smartdiagram[bubble diagram]{
		\textbf{\en{Engineering}\spa{Ingeniería}}\\\textbf{\en{Physics}\spa{Física}},
		\textbf{\en{Mathematical}\spa{Abstracción}}\\\textbf{\en{Abstraction}\spa{Matemática}},
		\textbf{\en{Automated}\spa{Conducción}}\\\textbf{\en{Driving}\spa{Autónoma}},
		\textbf{Problem}\\\textbf{Solving},
		\textbf{Machine}\\\textbf{Learning},
		\textbf{\en{Software}}\spa{Desarrollo}\\\textbf{\en{Development}\spa{Software}},
		\textbf{Computer}\\\textbf{Vision}
	}
}

% Programming skill bars
\programming{{HTML5 $\textbullet{}$ Java/ 1},{gnuplot $\textbullet{}$
			javascript/ 2},{Matlab $\textbullet{}$ ADTF $\textbullet{}$ CUDA
			$\textbullet{}$ OpenMP/ 3.5},{C $\textbullet{}$ C++  $\textbullet{}$ \LaTeX{} $\textbullet{}$ python  $\textbullet{}$ Mathematica/ 5}}

\languages{{\en{French}\spa{Francés}$\textbullet{}$
			\en{Polish}\spa{Polaco} /1},{\en{Spanish}\spa{Castellano}/
			5},{\en{Portuguese}\spa{Portugués} / 6},{\en{English}\spa{Inglés} / 6}}

----------------------------------------------------------------------------------------
%	 PERSONAL INFORMATION
%----------------------------------------------------------------------------------------

\cvname{Rui Marques}
\cvjobtitle{\en{Software Engineer}\spa{Ingeniero de Software}}
\profilepic{photo.png}
\cvlinkedin{in/ruiferreiramarques}
\cvgithub{github.com/ruifm}
\cvnumberphone{\href{tel:+34663559607}{(+34) 663 559 607}}
\cvsite{ruimarques.xyz}
\cvmail{inquiries@ruimarques.xyz}
\cvskype{ruifm75969}
\cvaddress{\href{https://goo.gl/maps/ikvtH4ZwUw6LTwAL8}{A Coruña, Spain}}
\cvtwitter{rui\_f\_marques}


\aboutme{
	\en{Deep Learning, Computer Vision and C++/python developer with an academic background in Theoretical Physics and Engineering.

		I am able to cope and adapt to new
		and challenging tasks. My main motivation is tackling interesting
		problems. I consider myself reasonably good self-learner.}
	\spa{Desarrollador especializado en Deep Learning, Visión Computacional
		y C++ con formación académica en Física Teórica y Ingeriría.

		Soy capaz de hacer frente y adaptarme a tareas nuevas y desafiantes. Mi
		principal motivación es abordar problemas interesantes. Me considero un
		\emph{self-learner} razonablemente bueno.}
}

%----------------------------------------------------------------------------------------

\begin{document}

\makeprofile{}

%----------------------------------------------------------------------------------------
%    EXPERIENCE
%----------------------------------------------------------------------------------------

\en{\section{Experience}}\spa{\section{Experiencia}}

\begin{twenty} % Environment for a list with descriptions
	\twentyitem
	{\en{April}\spa{Abril} 2018}
	{\en{Present}\spa{Presente}}
	{\en{Deep Learning ADAS Software Engineer}\spa{Ingeniero de Deep Learning y
			Software de ADAS}}
	{\href{https://www.xesolinnovation.com/}{Xesol Innovation}}
	{Vigo, \en{Spain}\spa{España}}
	{\begin{itemize}
			\item \en{Kept up to date with detection and segmentation
				      state-of-the-art Convolutional Neural Networks.}\spa{Busqué las
				      mejores redes neuronales convolucionales para detección y
				      segmentación.}
			\item\en{Managed internal and public datasets, their synthetic
				      augmentation, training, hyperparameter tuning, validation and
				      model exporting to different frameworks.}\spa{Gestioné datasets
				      internos y públicos, su augmentación sintética, entrenamiento,
				      ajuste de híperparametros, validación y exportación de modelos a otras
				      frameworks.}
			\item\en{Integration of deep learning models in C++ for  real-time
				      commercial applications with severe hardware
				      constraints.}\spa{Integración de modelos de deep learning en C++
				      para aplicaciones comerciales en tiempo real com grandes restricciones
				      de hardware.}
			\item \en{Developed a perception and fusion logic layer from scratch that
				      leverage data from neural networks and other software modules. Implemented an
				      Extended Kalman Filter to get a smooth and reliable representation of the
				      environment.}\spa{Desarrollé una capa de percepción y fusion desde el cero que
				      aprovecha los datos de redes neuronales y otros módulos de software. Implementé
				      un Extended Kalman Filter para obtener una representación estable y fiable del
				      entorno exterior.}
			\item\en{Managed and maintained the entire core of the ADAS
				      software.}\spa{Gestioné y mantuve todo el núcleo de software de ADAS.}
			\item \textbf{\en{Tools}\spa{Herramientas}:} C++, python, tensorflow, tensorflow C++
			      API, CLion, caffe, tf-slim, keras, OpenCV, CUDA, matplotlib,
			      numpy, bash, git, doxygen, cmake, Makefile, redmine, \LaTeX{}
		\end{itemize}}
	\twentyitem
	{\en{Jan}\spa{Ene} 2018}
	{Mar 2018}
	{\en{ADAS Software Developer}\spa{Desarrollador de Software de ADAS}}
	{\href{https://www.ctag.com/}{CTAG}}
	{Vigo, \en{Spain}\spa{España}}
	{\begin{itemize}
			\item\en{Worked in the beginning in an internal project as a LiDAR preprocessing developer for a self driving car.}\spa{Trabajé al principio en un proyecto interno como desarrollador del sistema LiDAR de un coche autónomo.}
			\item \en{Due to my good performance and quick learning, the company transfered me to a special task force that works directly to a client.
			      }\spa{Debido a la rapidez con que terminé el proyecto, me transferirán para un equipo que trabaja directamente para un cliente.}
			      % \item \en{Worked with 2 Agile Methodologies: SCRUM and Kanban.
			      % }\spa{Trabajé con 2 Metodologías Ágiles: SCRUM y Kanban.}
			      % \item \en{Developed software in a collaborative team environment.
			      % }\spa{Desarrollé software en un entorno colaborativo de equipo.}
			\item \en{Picked up the entire code base from a departing former employee and build up on that.
			      }\spa{Cogí la base de código entera de un compañero que estaba de salida.}
			\item \en{Managed, proposed and reviewed System and Software Requirements.
			      }\spa{Gestioné, propuse y revisé Requisitos de Sistema y de Software.}
			\item \en{Developed and maintained software at the Client's request always with quality and before the requested deadlines.
			      }\spa{Desarrollé y mantuve software cuando pedido por el Cliente siempre con cualidad y antes del plazo.}
			\item \en{Kept an efficient communication with the Client.
			      }\spa{Mantuve una comunicación eficiente con el Cliente.}
			\item \en{Performed weekly software verifications: sanity checks, runtime checks, code coverage tests, static tests (code linting against MISRA-C++) and reported them to the Client.  }\spa{Realicé verificaciones semanales de software: sanity checks, runtime checks, code coverage tests, static tests (code linting against MISRA-C++) y las reporté al Cliente.}
			\item \textbf{\en{Tools}\spa{Herramientas}:} C++, ADTF, Eclipse, Qt, python,
			      matplotlib, numpy, bash, Matlab, redmine, git, Octave, MSVS,
			      doxygen, cmake, CANAlyzer, CANoe, Wireshark, Makefile, JIRA,
			      Serena, DOORS, Google Docs, html, xml, \LaTeX{}
			\item \textbf{\en{Attended Courses}\spa{Formaciones}:} \en{V Life
				      Cycle, Management of Requirements}\spa{Ciclo de Vida V, Gestión
				      de Requisitos}, Automotive SPICE, ISO26262, DOORS, SCRUM
		\end{itemize}}
	% \twentyitem
	% {Sep 2017}
	% {Oct 2017}
	% {Commercial Sales Representative}
	% {M\&S Servicios y Marketing}
	% {A Coruña, \en{Spain}\spa{España}}
	% {\en{Responsible to bring a constant flow of profitable new business in the energy sector by providing state-of-the-art solutions to potential customers.}\spa{Responsable de traer un flujo constante de nuevos negocios rentables en el sector de la energía proporcionando soluciones de vanguardia a potenciales clientes.}}
	% \twentyitem
	% {\en{Aug}\spa{Ago} 2012}
	% {Sep 2012}
	% {\en{Particle Physics Intern}\spa{Becario de Física de Partículas}}
	% {\href{https://www.lip.pt}{LIP (\en{Portuguese Particle Physics Laboratory}\spa{Laboratório Portugués de Física de Partículas})}}
	% {\en{Lisbon}\spa{Lisboa}, Portugal}
	% {\en{Worked on cosmic ray data and computer simulations from the \emph{Pierre Auger Observatory}, Advisor}\spa{Trabajé en datos de rayos cósmicos y simulaciones por ordenador del \emph{Observatorio Pierre Auger}. Supervisor}: Pedro Abreu (\href{mailto:abreu@lip.pt}{abreu@lip.pt})
	% }
\end{twenty}

% \en{\section{Research}}\spa{\section{Investigación}}
% \begin{twenty}
% 	\twentyitem
% 	{Feb 2017}
% 	{\en{Jan}\spa{Ene} 2018}
% 	{MSc. \en{Research Assistant}\spa{Investigador Asistente}}
% 	{\href{https://centra.tecnico.ulisboa.pt/network/grit/team/}{GRIT-CENTRA}}
% 	{\en{Lisbon}\spa{Lisboa}, Portugal}
% 	{
% 		\en{\textbf{Thesis}: Massive Graviton Theories and vDVZ discontinuities
% 		}\spa{\textbf{Tesis}: Gravitones masivos y discontinuidades vDVZ}
% 		\begin{itemize}
% 			\item \en{Worked on an extension of Einstein's General Relativity in which the graviton has a non zero mass and worked out its possible implications to experimental results such as gravitational waves.}\spa{Investigué una extensión de la Teoría de Relatividad General de Einstein en que el gravitón tiene una masa no nula y exploré sus consequências en resultados experimentales como ondas gravitacionales.}
% 			\item \textbf{\en{Tools}\spa{Herramientas}}: Python, scikit-learn, \LaTeX{}, matplotlib, Mathematica, gnuplot
% 		\end{itemize}
% 	}
% \end{twenty}

%----------------------------------------------------------------------------------------
%	 EDUCATION
%----------------------------------------------------------------------------------------
\en{\section{Education}}\spa{\section{Formación Académica}}
\begin{twenty} % Environment for a list with descriptions
	\twentyitem
	{Sep 2016}
	{Jun 2017}
	{Erasmus \@ \textsc{\en{Theoretical Physics}\spa{Física Teórica}}}
	{\href{http://web.science.uu.nl/itf/}{ITF, \en{Utrecht University}\spa{Universidad de Utrecht}}}
	{Utrecht, \en{Netherlands}\spa{Holanda}}
	{\en{Studied Quantum Information and Cosmology topics.
			Academic Referee}\spa{Estudié tópicos de información cuántica.
			Referencia Académica}: Enrico Pajer
		(\href{mailto:e.pajer@uu.nl}{e.pajer@uu.nl}) | \normalsize
		\textsc{\en{GPA}\spa{Media}}: 8/10}
	\twentyitem
	{Sep 2012}
	{Feb 2018}
	{MSc in
		\textsc{\href{https://www.youtube.com/watch?v=0eoa0f5nVA0}{\en{Engineering
					Physics}\spa{Ingeniería Física}}} \faYoutubePlay}
	{\href{https://www.youtube.com/watch?v=EGue8EwE3mI}{Instituto Superior Técnico} \faYoutubePlay}
	{\en{Lisbon}\spa{Lisboa}, Portugal}
	{\en{High Energy Theoretical Physics specialization}\spa{Especialización en
			Física de Alta Energía} \\
		\en{\textbf{Thesis}: Massive Graviton Theories and vDVZ discontinuities
		}\spa{\textbf{Tesis}: Gravitones masivos y discontinuidades vDVZ}
		\begin{itemize}
			\item \en{Worked on an extension of Einstein's General Relativity in which the graviton has a non zero mass and worked out its possible implications to experimental results such as gravitational waves.}\spa{Investigué una extensión de la Teoría de Relatividad General de Einstein en que el gravitón tiene una masa no nula y exploré sus consequências en resultados experimentales como ondas gravitacionales.}
			\item \textbf{\en{Tools}\spa{Herramientas}}: Python, scikit-learn, \LaTeX{}, matplotlib, Mathematica, gnuplot
		\end{itemize}
	}
	% \twentyitem
	% {Sep 2009}
	% {Feb 2012}
	% {High School \@ \textsc{\en{Natural Sciences}\spa{Ciencias Naturales}}}
	% {Escola Secundária D. Pedro V}
	% {Lisbon, Portugal}
	% {\en{Physics and Math}\spa{Física y Matemática}: 20/20,
	% 	\en{English}\spa{Inglés} 19/20 | \en{2012 top student}\spa{mejor estudiante
	% 		de 2012}}
\end{twenty}

\newpage
\continuesidebar{}


\en{\section{Scholarships \& Certificates}}\spa{\section{Becas y Certificados}}
\begin{twenty} % Environment for a list with descriptions
	\twentyitem
	{Oct 2017}
	{}
	{\en{Neural Networks and Deep Learning by deeplearning.ai}\spa{Redes Neuronales y Aprendizaje Profundo por deeplearning.ai}}
	{\href{https://www.coursera.org/account/accomplishments/records/CF342FNNVUXY}{Coursera}}
	{}
	{}
	\twentyitem
	{Oct 2017}
	{}
	{\en{Improving Deep Neural Networks: Hyperparameter tuning, Regularization
			and Optimization by deeplearning.ai}\spa{Mejora de Redes Neuronales
			Profundas: Optimización, Regularización y Optimización de Hiperparametros por deeplearning.ai}}
	{\href{https://www.coursera.org/account/accomplishments/records/MQVCJFJ849GK}{Coursera}}
	{}
	{}
	\twentyitem
	{May 2012}
	{}
	{\en{Nacional Finalist}\spa{Finalista Nacional}}
	{\href{http://www.sp-astronomia.pt/olimpiadas}{\en{Astronomy and Astrophysics Portuguese Olympiads}\spa{Olimpiadas de Astronomía e Astrofísica Portuguesas}}}
	{São Miguel, Azores, Portugal}
	{}
	\twentyitem
	{Jan 2012}
	{May 2012}
	{\en{Advanced precollege physics school}\spa{Escuela de física pre-universitaria avanzada}}
	{\href{http://quark.fis.uc.pt/}{\en{Quark! Project}\spa{Proyecto Quark!}}}
	{Coimbra, Portugal}
	{}
\end{twenty}

\en{\section{Programming Projects}}\spa{\section{Proyectos de Programación}}

\begin{twenty} % Environment for a list with descriptions
	\twentyitem
	{Oct 2017}
	{Nov 2017}
	{Space Out}
	{\href{https://github.com/ruifm/space-out}{\faGithub github.com/ruifm/space-out}}
	{}
	{\en{pygame 1 vs 1 version of the classic arcade game `Asteroids'. Intended for training of a reinforcement learning neural network to work as an AI opponent.}\spa{Versión 1 vs 1 en Pygame del clásico juego de arcade `Asteroids'. Destinado para el entrenamiento de una red neuronal de refuerzo de aprendizaje para trabajar como un oponente de IA.}}
	\twentyitem
	{Dec 2016}
	{\en{Jan}\spa{Ene} 2017}
	{\en{Entanglement Entropy}\spa{Entropia de Enredo}}
	{\href{https://github.com/ruifm/ed-triangular}{\faGithub github.com/ruifm/ed-triangular}}
	{}
	{\en{Mathematica notebook for the exact digitalization and Entanglement Entropy computation in a Triangular Spin lattice.}\spa{Cuaderno de Mathematica para la diagonalización exacta y cálculo de entropía de enredo en un enrejado de spin triangular.}}
	\twentyitem
	{May 2015}
	{}
	{Oort Cloud}
	{\href{https://github.com/ruifm/oort}{\faGithub github.com/ruifm/oort}}
	{}
	{\en{A javascript version using the Phaser framework of the classic arcade game
			`Asteroids', modified by giving it a little bit of a `flappy bird' tone,
			i.e.\ a never ending game. Hosted here:
			\href{www.xente.mundo-r.com/20624313W0001/index.html}{xente.mundo-r.com/20624313W0001/index.html}}\spa{Una
			versión de JavaScript que utiliza el framework Phaser del clásico juego de
			arcade `Asteroids', modificado al darle un toque de `flappy bird', i.e.\ un juego interminable. Alojado aquí: \href{www.xente.mundo-r.com/20624313W0001/index.html}{xente.mundo-r.com/20624313W0001/index.html}}}
	\twentyitem
	{May 2014}
	{}
	{Antifitter}
	{\href{https://github.com/ruifm/antifitter}{\faGithub github.com/ruifm/antifitter}}
	{}
	{\en{C++ program that yields fake experimental data to fit a certain function and plots it automatically.}\spa{Programa en C ++ que produce datos experimentales falsos para ajustarse a una determinada función y traza automáticamente.}}
	\twentyitem
	{Dec 2013}
	{\en{Jan}\spa{Ene} 2014}
	{Gross-Pitaevskii Simulator}
	{\href{https://github.com/ruifm/gross-pitaevskii}{\faGithub github.com/ruifm/gross-pitaevskii}}
	{}
	{\en{Color density plot simulation of the Gross-Pitaevskii equation applied to
			a Bose-Einstein Condensate using C++, OpenMP and CUDA\@. Result:
			\href{https://youtu.be/V091IqIRV4c}{youtu.be/V091IqIRV4c}}\spa{Simulación
			de la gráfica de densidad de color de la ecuación de Gross-Pitaevskii
			aplicada a un condensado de Bose-Einstein usando C++, OpenMP y CUDA\@. Resultado: \href{https://youtu.be/V091IqIRV4c}{youtu.be/V091IqIRV4c}}}
	\twentyitem
	{Dec 2012}
	{\en{Jan}\spa{Ene} 2013}
	{Coloumbian Simulator}
	{\href{https://github.com/ruifm/charges}{\faGithub github.com/ruifm/charges}}
	{}
	{\en{Coulomb force in charges simulator written in C with GTK+.}\spa{Simulador de la fuerza de Coulomb en cargos escrito en C con GTK+.}}
	\twentyitem
	{Nov 2012}
	{}
	{Atkin's Sieve in C}
	{\href{https://github.com/ruifm/atkin}{\faGithub github.com/ruifm/atkin}}
	{}
	{\en{Atkin implementation for finding prime numbers in C with many features.}\spa{Implementación de Atkin para encontrar números primos en C con muchas características.}}
\end{twenty}

\en{\section{Projects and Affiliations}}\spa{\section{Proyectos y Afiliaciones}}
\begin{twenty}
	\twentyitem
	{Feb 2014}
	{}
	{\en{Co-host and Organizer}\spa{Co-anfirión y organizador}}
	{\href{http://jornadasdefisica.nfist.pt/index.html}{\en{Engineering Physics Days}\spa{Jornadas de Ingeniería Física}}}
	{\en{Lisbon}\spa{Lisboa}, Portugal}
	{\en{A 3 day event with physics and engineering seminars from researchers and potential employers. I led a 7 people task force to put together this amazing event.}\spa{Un evento de 3 días con seminarios de física e ingeniería con investigadores y potenciales empleadores. Dirigí un grupo de trabajo de 7 personas para organizar este asombroso evento.}}
	\twentyitem
	{Sep 2013}
	{Sep 2014}
	{\en{Science entertainer}\spa{Animador de Ciencias}}
	{\href{http://festadoavante.pcp.pt/2016/}{Festa do Avante}}
	{Amora, \en{Lisbon}\spa{Lisboa}, Portugal}
	{\en{Performed live physics experiments to the general public during the \emph{Avante} Summer festival.}\spa{He realizado experimentos de física en vivo al público durante el festival de verano del \emph{Avante}.}}
	\twentyitem
	{2012}
	{2015}
	{\en{Board Member}\spa{Miembro de la Junta Directiva}}
	{\href{http://nfist.pt}{NFIST} member (IST physics club)}
	{\en{Lisbon}\spa{Lisboa}, Portugal}
	{\en{A dynamic and productive non-profitable organization with tremendous scientific outreach. It's main goal is to create public awareness to the beauty and omnipresence of physics in our daily lives. I've been and active science speaker in public schools, museums and other events.}\spa{Una organización dinámica y productiva sin fines de lucro con un enorme
			alcance científico. Su principal objetivo es crear conciencia pública sobre la belleza y la omnipresencia de la física en nuestra vida cotidiana. Enseñé activamente ciencias en escuelas públicas, museos y y otros eventos.}}
	\twentyitem
	{Sep 2012}
	{}
	{\en{Staff member}\spa{Miembro del personal}}
	{\href{http://www.lip.pt/cmsweek2012/}{CMS Week 2012}}
	{\en{Lisbon}\spa{Lisboa}, Portugal}
	{\en{I was invited by a LIP physicist (former advisor) to be a part of the organizing committee responsible for hosting the event. I had the opportunity to meet renowned physicists worldwide.}\spa{Un físico de LIP (ex supervisor) me invitó a formar parte del comité organizador y responsable de organizar el evento. Tuve la oportunidad de conocer físicos de renombre en todo el mundo.}}
\end{twenty}

\end{document}
